
\chapter{Introduction}\label{sec:intro}
    \pagenumbering{arabic}
    

    \section{Recommender systems}
    
        Every year the amount of data created increases exponentially. Included in this increase is data such as movies and other online content. There is more and more media available to watch but the amount of time people have to spend watching this content is not increasing. Every year it becomes easier and easier for individuals to create their media content and share it on the internet. In 2001, 356 movies were released between the United States and Canada. The number of movies released each year has been increasing since then. The peak number was in 2018 when 878 movies were released. This is over a doubling since 2001. \cite{NumberOfMoviesReleased} This equates to two movies a day for an entire year. Most of these movies will not interest everyone, this problem can be mitigated by using movie recommenders. 


        The idea of movie recommenders is not new. Online streaming sites often include recommendation systems to help the user find the movies they want to quickly. Amazon and Netflix are the biggest movie streaming website and both have built-in recommendation engines that help users browse their websites. They do this because they have a huge backlog of content much of which will not be relevant to many people so as to not alienate customers they attempt to steer them in the right direction. When using these sites users get a curated selection of movies given to them. Sometimes these come with explanations as to why they should be chosen such as "We think you will like movie X as you have seen movie Y before. 
        They will also show you movie genres that are common between movies you have seen recently. The explanations for these movies recommendations rarely get more complex than if you like this movie you will like another. Although most streaming services contain a recommender I have not heard of a cinema that offers such services. In this project, we  plan to implement a recommender that will take movies from a local cinema and use them for recommendations. We will also explore creating a Netflix like recommender where we use more movies that just ones that are a local cinema.

    \section{Explanations}
        We want to solve the problem of a recommender not giving detailed enough explanations along with its recommendations. We want to explore creating a system that creates more explanations for a recommendation than, you might like X because you like Y. 
        It is assumed this is the case. We are going to use the assumption that if you like a movie and another user who like that movie might share other movies that you both might like. Therefore creating recommendations that we can show you. We can then use this movie and information about similar movies to create explanations tailored to you. We limited the project to recommendations based on movie ratings a user would provide the system. 


    \section{Balloon Debate}
        
        From our research, it appears seems that the concept of a recommender in the form of a debate has not been implemented. Where one speaker is removed per round. This gives the user some immediacy when playing the game as they feel they have to make a choice. This seems to be a novel approach that could give users an interesting experience while using a recommender. In our case the user  
        will have a selection of movies that are recommended to them. Each round they will be shown an explanation as to why they should keep that movie in the debate. After the user picks a movie to remove the next round begins and new explanations are shown. This continues until the user has only one movie remaining. There was an example of a recommender that took information created from a debate and used that to create recommendations for the user's friends \cite{10.1145/2792838.2799675}. This was similar to the planned project in  that it was a recommender that was more involved than picking a movie and explaining the reasons with other movies but ours will give recommendations in the game as apposed to using it to harvest data. Although what we are creating is a separate system that forms a game for a user to play. Pieces of the system could bae taken and implemented into a movie streaming site where movies are recommended and then explained using the above explanations. We  chose to structure the debate between six movies as that seemed like a good number where too few would not give enough rounds 


    \section{Project Overview}
        The aim was to accommodate a balloon debate between a set of movies where the user is the adjudicator. Movies currently being shown in the local cinema would compete against each other to be chosen by the user to see. Movies would give explanations as to why they should be picked and a user would be able to remove after each round of the debate. A system was created to do just this. The extent to which the movies can explain themselves was not decided in the outline but the system was extensible and more could be added. 
        
        In this chapter, we provided an introduction to the project and some background on why it is needed. In chapter 2 we will discuss some more background reading that is required for the project about recommender systems and about the work going into explaining the models created by artificial intelligence. We will also discuss the requirements for this project and what we hope to complete. In chapter 3 we are going to explore the creation of the recommender systems that were made in this project. In chapter 4 we are going to talk about the evaluation of the system where we created a user trial and invited people to test out the system and the results from the user testing. In the final chapter, we will talk about the conclusion of the project and some future work that could be undertaken. 